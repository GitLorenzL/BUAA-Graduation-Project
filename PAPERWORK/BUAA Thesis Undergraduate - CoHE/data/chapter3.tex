% !Mode:: "TeX:UTF-8"
\chapter{方法设计与实现}

\section{方法概述(Pipeline)}

\begin{enumerate}
	\item 场景浏览与目标选择;
	\item 操纵模式选择。生成一个“四叶草”模式选择菜单,用户可以通过凝视某一选项以进入对应的操纵模式,用户也可以通过凝视“返回”选项以退回到1(详见\autoref{Clover});
	\item 对象操纵:位移、旋转、缩放(详见\autoref{Manipulation});
	\item 确认操纵结果,回到3。
\end{enumerate}

\section{场景浏览与目标选择}

\section{“四叶草”模式选择菜单}\label{Clover}

\subsection{设计理念}

我们在交互系统设计阶段考虑到大多数实际的完整操纵流程往往不仅包含一个特定操纵模式的选择,还会包含多种操纵模式之间的切换。然而,当前的基于凝视和眼动的交互方法并没有考虑便捷的模式切换;在这些方法中,如果用户需要在一次操纵结束后切换到下一种操纵模式,则需要退回到最初态重新进行一遍从选择到操纵到流程,引入大量冗杂操纵和效率浪费。所以基于此考虑,我们设计了一个“四叶草”模式选择菜单。

我们希望“四叶草”模式选择菜单可以让用户简便地用凝视动作来选择进入或者切换到某种交互状态。

如果用户需要在菜单中选择某一选项


\subsection{菜单说明}

\subsection{线性优化}

\section{对象操纵}\label{Manipulation}
