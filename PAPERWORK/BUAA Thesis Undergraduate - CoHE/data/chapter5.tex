% !Mode:: "TeX:UTF-8"
\chapter{结论}

\section{结论与展望}

本论文提出了一种在虚拟现实中基于头眼协同的对象操作方法。该方法是凝视主导且完全无手干预的,实现了包括空间平移、旋转和等比例缩放的完全6DOF操作。我们还提出了一个完整的方法流程和一个基于3D用户界面的“四叶草”模式选择菜单,让用户只需用眼动信号即可轻松切换不同的操作模式,实现了在虚拟空间中对多对象的连续交互操作。为应对眼动追踪数据嘈杂的问题,我们引入了过滤算法和线性优化过程来增强用户体验,确保交互自然流畅。这种新型交互系统使身体有障碍的人或手部被占用或限制的场景中的物体操作变得更加容易。通过消除对手部的需求,这种方法为与虚拟现实环境的交互提供了一种新的方式,扩大了沉浸式体验的可用性。

综上,本次研究的主要贡献在于:

\begin{itemize}
	\item 提出了一个基于头眼协同的、完全无手干预的对象操纵方法,实现了空间位移、空间定轴旋转和空间等比例缩放以及各操纵模式之间的快速切换,并且显著优于当前基于头眼协同的最优对象操纵方法;
	\item 提出了一个“四叶草”模式选择菜单,解决了对象操纵研究的模式间切换问题,构建了一个完整的闭环操纵流程;
	\item 提出了一种眼动信号的滤波算法和一种头眼协同的凝视驻留点计算优化;
	\item 提出了一种全面测试对象操纵的“积木”用户实验,有助于相关研究对其方法效果的定量测试。
\end{itemize}

\section{方法局限性}

我们的方法主要有以下三点局限性:
\begin{enumerate}
	\item 我们对于该交互系统的预设为用户始终固定在一个位置进行操纵,所以在我们的头眼信号-对象行为的映射关系中没有考虑用户位置移动所带来的影响;
	\item 用户在使用眼动信号进行交互时难以做到随时观察对象的状态,导致在很多情况下需要进行多次调整,这也是系统可用性损失的最大部分;
	\item 我们在目标对象选择时没有考虑遮挡问题,所以我们的交互系统暂时无法快速处理目标对象被某一障碍物遮挡的情况。
\end{enumerate}

\section{后续工作}

针对后续可开展的工作,我们首先可以针对方法的局限性进行优化和完善。为了解决固定映射影响用户体验的问题,我们可以引入相对坐标系的概念以根据用户位置而相对地生成头眼信号-对象行为的映射关系。为了解决眼动操纵时无法观察对象的问题,我们可以在目标对象处设置一个定点虚拟摄像头,并且在用户的凝视射线的一定距离处添加一个可视窗口,以实时反馈目标对象的状态。为了解决选择对象时的遮挡问题,一个假想为设置一个预选集,其内容为用户凝视射线穿透的所有物体,供用户进行二次选择。

除此之外,我们也可以在后续的工作中创建并完善该对象操纵方法的应用程序接口(API),允许更多的应用程序接入并使用该操纵方法,以促进更新与迭代,从而增强该操纵方法的功能和价值。