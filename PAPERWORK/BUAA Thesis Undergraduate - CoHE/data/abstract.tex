% !Mode:: "TeX:UTF-8"

% 中英文摘要
\begin{cabstract}
本论文提出了一种在虚拟现实中基于头眼协同的对象操作方法。该方法是凝视主导且完全无手干预的,实现了包括空间平移、旋转和等比例缩放的完全6DOF操作。我们还提出了一个完整的方法流程和一个基于3D用户界面的“四叶草”模式选择菜单,让用户只需用眼动信号即可轻松切换不同的操作模式,实现了在虚拟空间中对多对象的连续交互操作。为应对眼动追踪数据的高噪音问题,我们引入了过滤算法和线性优化过程来增强用户体验,确保交互自然流畅。这种新型交互系统使身体有障碍的人或手部被占用或限制的场景中的物体操作变得更加容易。通过消除对手部的需求,这种方法为与虚拟现实环境的交互提供了一种新的方式,扩大了沉浸式体验的可用性。
\end{cabstract}

\begin{eabstract}
This paper proposes a head-eye collaborative object manipulation in virtual reality. Our eye-dominant and hands-free method enables spatial translation, rotation, and proportional scaling with a complete 6DOF manipulation. We also realize an entire pipeline with a mode-switching 3D user interface menu called ``Clover," allowing users to effortlessly switch between different manipulation modes with sole eye movement. The pipeline enables sequential and incessant interaction actions with multiple objects in a virtual environment. To address the challenges posed by noisy eye-tracking data, we introduce a filtering algorithm and a linear optimization process to enhance the user experience, ensuring that the interaction is both natural and fluent. This novel interaction system makes object manipulation accessible to individuals with physical disabilities or in scenarios where hands are either occupied or restrained. By eliminating the need for hands, this approach provides a new way of interacting with virtual reality environments, expanding the usability of immersive experiences.
\end{eabstract}